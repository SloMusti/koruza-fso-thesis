\chapter{Introductory Material}
\label{chapterlabel1}

The Number of Interconnected Devices Surpassed the World’s Population: The increasing array of digital services ranging from different fields such Health Care, Transportation, Culture, Leisure and Sports, has reshaped our patterns of consumption and needs. Managers and directors have also observed this Digital Revolution in their professional activities, since new paradigms have been established in sectors such as Remote Sensing and Security. The easy-to-access and easy-to-use technological layout is continuously boosting the delivery of more and better smartphones, tablets, computers, TV, medical devices and sensors. For example, just in 2011 the number of interconnected devices on the planet surpassed the actual number of people. Currently, there are 9 billion interconnected devices and it is expected to reach 24 billion devices by 2020 \cite{RePEc:eee:bushor:v:58:y:2015:i:4:p:431-440}.

Internet of Things, Cloud Computing and Virtual \& Augmented Reality are Growing Exponentially: Either to support or enhance the user’s experience, the increased number of interconnected devices rely on services and applications based on the Internet of Things (IoT), Cloud Computing and Virtual \& Augmented Reality (VR/AR). For example, McKinsey estimates the total IoT market size in 2015 was up to USD900 million, growing to USD3.7 thousand million in 2020 attaining a 32.6\% CAGR.[2] On the other hand, cloud computing is projected to increase from USD67 thousand million in 2015 to USD162 thousand million in 2020 attaining a CAGR of 19\%.[3] Last but not least, in 2021, the projected combined market for VR \& AR is USD108 thousand million.[4] Data shows that fast and stable access to the Internet is therefore becoming an important factor, which is impacting the development of a digital economy and the growth of economy. That is why all over the world, the task of connecting homes and businesses with broadband connections is on the public agenda.[5]

We Have the Devices, But We Do Not Have a Solid and Cheap High-Speed Network for Them: Despite current efforts for novel devices and turning them compliant with Internet of Things, Cloud Computing and Virtual \& Augmented Reality best practices, uninterrupted 5G coverage for urban areas and major terrestrial transport paths or access to connectivity offering at least 100 Mbps in households’ environments, are still nowadays hard to provide. For example, Europe is lagging in those metrics as recognized by the European Commission.[5] Supporting high capacity Internet connection to end users usually comprises two problems: i) high costs in hardware and ii) complex installation procedures (e.g. reaching a household from the fibre cable end point). Moreover, the data flux is also increasingly daily, making current Internet providing solutions obsolete or too expensive for the consumers. For example, in 2014, out of the 42 exabytes per month of consumer Internet traffic, 56\% were due to Internet video.[6]

% References to be inserted: 1Lee et al., Business Horizons (2015) 58:431-440; 2Mc Kinzey & Company, Internet of Things: The IoT opportunity (2016); 3Salesforce, The Salesforce Economy (2016); 4Digi-Capital AR/VR Report Q3 2017; 5European Commission, Connectivity for a European Gigabit Society (2016); 6EC FIArch Group, Fundamental Limitations of current Internet and the path to Future Internet (2011)


Solution: KORUZA (www.koruza.net) is a license-free free-space optical wireless Internet access system for last-mile applications, which uses an eye-safe collimated beam of InfraRed-light to securely transmit data point-to-point through air. Our solution avoids digging of the roads, allowing distances up to 150 m with fibre-like speeds (1 Gbps or 10 Gbps). KORUZA is the world’s first affordable free-space optical system, dramatically reducing its cost by combining mass-produced parts, best-practice networking solutions and open standards.

KORUZA costs no more than EUR10/month, turning our solution 10 times cheaper in comparison with standard fibre-based solutions. Our technology does not suffer from radio frequency interference or band over-crowdedness, being immune to electromagnetic interferences.  The technology is also a license-, regulation- and permission-free networking solution. KORUZA transmits data with light between buildings, and its eye-safe operation is governed by health-and-safety regulations. 
KORUZA low-latency and low-jitter turns it suitable not only for broadband subscribers, but also for 4G and 5G networks. It can be installed in a matter of hours, deliver high-speed broadband without interference from other wireless systems. Our product is based in a modular construction, being flexible for technology modifications and integrations regarding specially uses, as well as it comprises a possible integration with other devices to provide alternative Internet infrastructures. Moreover, it is low-maintenance and operates an active link-protection

\section{Problem statement}
Current methods available to provide broadband connectivity are DSL, VDSL, licensed RF (e.g. WiMax), Cable, Docsis 3.0, unlicensed RF (e.g. WiFi), HSPA, LTE and Satellite. However, only fibre and RF enable high-speed connectivity. Installing cables comes with high costs of trenching, being often complicated because of permits-obtaining and legal-grounds-covering procedures. Even though high capacity fibre access networks are constantly being extended, many subscribers turn down the offer for a fibre based broadband plan, since the one-time installation is an expensive and inconvenient process, often requiring re-trenching and re-landscaping nearby areas. Whereas unlicensed RF seems to solve the need of complex and costly cable implementation, unlicensed RF’s band is over-crowded, especially in dense urban areas, causing radio frequency and electromagnetic interference and limiting the total available throughput for users in the area. Since the unlicensed spectrum is not unlimited and it is becoming overcrowded, such competitive solutions are not currently managing to deliver more 1 Gbps capacities even in ideal scenarios without interference. Globally speaking, as the fibre-based networking technology progresses, its impact is minimal since it usually requires complex and labour-intensive actions (e.g. full substitution of a fibre’s core/cable). On the other hand, radio-based technologies must be completely redesigned from a technical point-of-view to enhance their throughputs. Technologically speaking, commercially available solutions deploy limited solutions, where the consumer must choose between expensive services or poor Internet access/use, being frequently unserved in the need of future infrastructure updates.

Novelty and Added Value of KORUZA: As a state-of-the-art FSO Internet access system, KORUZA offers a solution for everyone in need of license-free operations with affordable, reliable and fast-deployed connections, for example, in smart cities scenarios providing an alternative internet access infrastructure, in radio astronomical observatories for management of radio telescopes, or in transportation and logistics centres for management of communication between harbour cranes, trains and similar. The technology is not affected by radio frequency interference or band over-crowdedness, being also immune to electromagnetic interferences. Regardless weather influence, KORUZA pilots’ data already strongly suggest a solid Internet access, high data flow and Internet speed, outperforming other practices in terms of costs, easy-to-install/easy-to-use, absent RF interference and modular architecture.

\section{Contributions}
This thesis makes the following contributions:

\paragraph{Analysis of the FSO market and the industry needs} to determine the requirements of future FSO systems, their use-cases and implementation requirements in telecommunication companies. The analysis is expanded to other industry fields, for example transportation and logistics sector, scientific research and infrastructure. 

\paragraph{Design of KORUZA wireless optical system} is the worlds first cost-effective gigabit+ wireless optical system for last-mile applications and leveraging open-source open-hardware best practices to create an industry-wide change in FSO system design.

\paragraph{World-wide FSO experiment} is the worlds largest planned experimental evaluation of FSO optical links to create a comprehensive data-set for the scientific analysis of the free-space channel with the focus on analyzing the hyper-local events on link performance.

\paragraph{Free-space optical channel link reliability model} correlating observable meteorological, environmental and structural parameters to FSO link performance, empowering the industry to reliably predict and understand these effects on future links to be deployed.

\paragraph{Implementation and development of open-source open-hardware practices} enables rapid dissemination and implementation of the technology advances developed in this project to the FSO industry.

\section{Thesis outline}
Explanation of chapter content. 

% This just dumps some pseudolatin in so you can see some text in place.
\blindtext
